\documentclass[11pt]{article}

    \usepackage[breakable]{tcolorbox}
    \usepackage{parskip} % Stop auto-indenting (to mimic markdown behaviour)
    
    \usepackage{iftex}
    \ifPDFTeX
    	\usepackage[T1]{fontenc}
    	\usepackage{mathpazo}
    \else
    	\usepackage{fontspec}
    \fi
    \usepackage[T2A]{fontenc}
    \usepackage[utf8]{inputenc}
    \usepackage[russian]{babel}

    % Basic figure setup, for now with no caption control since it's done
    % automatically by Pandoc (which extracts ![](path) syntax from Markdown).
    \usepackage{graphicx}
    % Maintain compatibility with old templates. Remove in nbconvert 6.0
    \let\Oldincludegraphics\includegraphics
    % Ensure that by default, figures have no caption (until we provide a
    % proper Figure object with a Caption API and a way to capture that
    % in the conversion process - todo).
    \usepackage{caption}
    \DeclareCaptionFormat{nocaption}{}
    \captionsetup{format=nocaption,aboveskip=0pt,belowskip=0pt}

    \usepackage{float}
    \floatplacement{figure}{H} % forces figures to be placed at the correct location
    \usepackage{xcolor} % Allow colors to be defined
    \usepackage{enumerate} % Needed for markdown enumerations to work
    \usepackage{geometry} % Used to adjust the document margins
    \usepackage{amsmath} % Equations
    \usepackage{amssymb} % Equations
    \usepackage{textcomp} % defines textquotesingle
    % Hack from http://tex.stackexchange.com/a/47451/13684:
    \AtBeginDocument{%
        \def\PYZsq{\textquotesingle}% Upright quotes in Pygmentized code
    }
    \usepackage{upquote} % Upright quotes for verbatim code
    \usepackage{eurosym} % defines \euro
    \usepackage[mathletters]{ucs} % Extended unicode (utf-8) support
    \usepackage{fancyvrb} % verbatim replacement that allows latex
    \usepackage{grffile} % extends the file name processing of package graphics 
                         % to support a larger range
    \makeatletter % fix for old versions of grffile with XeLaTeX
    \@ifpackagelater{grffile}{2019/11/01}
    {
      % Do nothing on new versions
    }
    {
      \def\Gread@@xetex#1{%
        \IfFileExists{"\Gin@base".bb}%
        {\Gread@eps{\Gin@base.bb}}%
        {\Gread@@xetex@aux#1}%
      }
    }
    \makeatother
    \usepackage[Export]{adjustbox} % Used to constrain images to a maximum size
    \adjustboxset{max size={0.9\linewidth}{0.9\paperheight}}

    % The hyperref package gives us a pdf with properly built
    % internal navigation ('pdf bookmarks' for the table of contents,
    % internal cross-reference links, web links for URLs, etc.)
    \usepackage{hyperref}
    % The default LaTeX title has an obnoxious amount of whitespace. By default,
    % titling removes some of it. It also provides customization options.
    \usepackage{titling}
    \usepackage{longtable} % longtable support required by pandoc >1.10
    \usepackage{booktabs}  % table support for pandoc > 1.12.2
    \usepackage[inline]{enumitem} % IRkernel/repr support (it uses the enumerate* environment)
    \usepackage[normalem]{ulem} % ulem is needed to support strikethroughs (\sout)
                                % normalem makes italics be italics, not underlines
    \usepackage{mathrsfs}
    

    
    % Colors for the hyperref package
    \definecolor{urlcolor}{rgb}{0,.145,.698}
    \definecolor{linkcolor}{rgb}{.71,0.21,0.01}
    \definecolor{citecolor}{rgb}{.12,.54,.11}

    % ANSI colors
    \definecolor{ansi-black}{HTML}{3E424D}
    \definecolor{ansi-black-intense}{HTML}{282C36}
    \definecolor{ansi-red}{HTML}{E75C58}
    \definecolor{ansi-red-intense}{HTML}{B22B31}
    \definecolor{ansi-green}{HTML}{00A250}
    \definecolor{ansi-green-intense}{HTML}{007427}
    \definecolor{ansi-yellow}{HTML}{DDB62B}
    \definecolor{ansi-yellow-intense}{HTML}{B27D12}
    \definecolor{ansi-blue}{HTML}{208FFB}
    \definecolor{ansi-blue-intense}{HTML}{0065CA}
    \definecolor{ansi-magenta}{HTML}{D160C4}
    \definecolor{ansi-magenta-intense}{HTML}{A03196}
    \definecolor{ansi-cyan}{HTML}{60C6C8}
    \definecolor{ansi-cyan-intense}{HTML}{258F8F}
    \definecolor{ansi-white}{HTML}{C5C1B4}
    \definecolor{ansi-white-intense}{HTML}{A1A6B2}
    \definecolor{ansi-default-inverse-fg}{HTML}{FFFFFF}
    \definecolor{ansi-default-inverse-bg}{HTML}{000000}

    % common color for the border for error outputs.
    \definecolor{outerrorbackground}{HTML}{FFDFDF}

    % commands and environments needed by pandoc snippets
    % extracted from the output of `pandoc -s`
    \providecommand{\tightlist}{%
      \setlength{\itemsep}{0pt}\setlength{\parskip}{0pt}}
    \DefineVerbatimEnvironment{Highlighting}{Verbatim}{commandchars=\\\{\}}
    % Add ',fontsize=\small' for more characters per line
    \newenvironment{Shaded}{}{}
    \newcommand{\KeywordTok}[1]{\textcolor[rgb]{0.00,0.44,0.13}{\textbf{{#1}}}}
    \newcommand{\DataTypeTok}[1]{\textcolor[rgb]{0.56,0.13,0.00}{{#1}}}
    \newcommand{\DecValTok}[1]{\textcolor[rgb]{0.25,0.63,0.44}{{#1}}}
    \newcommand{\BaseNTok}[1]{\textcolor[rgb]{0.25,0.63,0.44}{{#1}}}
    \newcommand{\FloatTok}[1]{\textcolor[rgb]{0.25,0.63,0.44}{{#1}}}
    \newcommand{\CharTok}[1]{\textcolor[rgb]{0.25,0.44,0.63}{{#1}}}
    \newcommand{\StringTok}[1]{\textcolor[rgb]{0.25,0.44,0.63}{{#1}}}
    \newcommand{\CommentTok}[1]{\textcolor[rgb]{0.38,0.63,0.69}{\textit{{#1}}}}
    \newcommand{\OtherTok}[1]{\textcolor[rgb]{0.00,0.44,0.13}{{#1}}}
    \newcommand{\AlertTok}[1]{\textcolor[rgb]{1.00,0.00,0.00}{\textbf{{#1}}}}
    \newcommand{\FunctionTok}[1]{\textcolor[rgb]{0.02,0.16,0.49}{{#1}}}
    \newcommand{\RegionMarkerTok}[1]{{#1}}
    \newcommand{\ErrorTok}[1]{\textcolor[rgb]{1.00,0.00,0.00}{\textbf{{#1}}}}
    \newcommand{\NormalTok}[1]{{#1}}
    
    % Additional commands for more recent versions of Pandoc
    \newcommand{\ConstantTok}[1]{\textcolor[rgb]{0.53,0.00,0.00}{{#1}}}
    \newcommand{\SpecialCharTok}[1]{\textcolor[rgb]{0.25,0.44,0.63}{{#1}}}
    \newcommand{\VerbatimStringTok}[1]{\textcolor[rgb]{0.25,0.44,0.63}{{#1}}}
    \newcommand{\SpecialStringTok}[1]{\textcolor[rgb]{0.73,0.40,0.53}{{#1}}}
    \newcommand{\ImportTok}[1]{{#1}}
    \newcommand{\DocumentationTok}[1]{\textcolor[rgb]{0.73,0.13,0.13}{\textit{{#1}}}}
    \newcommand{\AnnotationTok}[1]{\textcolor[rgb]{0.38,0.63,0.69}{\textbf{\textit{{#1}}}}}
    \newcommand{\CommentVarTok}[1]{\textcolor[rgb]{0.38,0.63,0.69}{\textbf{\textit{{#1}}}}}
    \newcommand{\VariableTok}[1]{\textcolor[rgb]{0.10,0.09,0.49}{{#1}}}
    \newcommand{\ControlFlowTok}[1]{\textcolor[rgb]{0.00,0.44,0.13}{\textbf{{#1}}}}
    \newcommand{\OperatorTok}[1]{\textcolor[rgb]{0.40,0.40,0.40}{{#1}}}
    \newcommand{\BuiltInTok}[1]{{#1}}
    \newcommand{\ExtensionTok}[1]{{#1}}
    \newcommand{\PreprocessorTok}[1]{\textcolor[rgb]{0.74,0.48,0.00}{{#1}}}
    \newcommand{\AttributeTok}[1]{\textcolor[rgb]{0.49,0.56,0.16}{{#1}}}
    \newcommand{\InformationTok}[1]{\textcolor[rgb]{0.38,0.63,0.69}{\textbf{\textit{{#1}}}}}
    \newcommand{\WarningTok}[1]{\textcolor[rgb]{0.38,0.63,0.69}{\textbf{\textit{{#1}}}}}
    
    
    % Define a nice break command that doesn't care if a line doesn't already
    % exist.
    \def\br{\hspace*{\fill} \\* }
    % Math Jax compatibility definitions
    \def\gt{>}
    \def\lt{<}
    \let\Oldtex\TeX
    \let\Oldlatex\LaTeX
    \renewcommand{\TeX}{\textrm{\Oldtex}}
    \renewcommand{\LaTeX}{\textrm{\Oldlatex}}
    % Document parameters
    % Document title
    \title{Lab3}
    
    
    
    
    
% Pygments definitions
\makeatletter
\def\PY@reset{\let\PY@it=\relax \let\PY@bf=\relax%
    \let\PY@ul=\relax \let\PY@tc=\relax%
    \let\PY@bc=\relax \let\PY@ff=\relax}
\def\PY@tok#1{\csname PY@tok@#1\endcsname}
\def\PY@toks#1+{\ifx\relax#1\empty\else%
    \PY@tok{#1}\expandafter\PY@toks\fi}
\def\PY@do#1{\PY@bc{\PY@tc{\PY@ul{%
    \PY@it{\PY@bf{\PY@ff{#1}}}}}}}
\def\PY#1#2{\PY@reset\PY@toks#1+\relax+\PY@do{#2}}

\@namedef{PY@tok@w}{\def\PY@tc##1{\textcolor[rgb]{0.73,0.73,0.73}{##1}}}
\@namedef{PY@tok@c}{\let\PY@it=\textit\def\PY@tc##1{\textcolor[rgb]{0.25,0.50,0.50}{##1}}}
\@namedef{PY@tok@cp}{\def\PY@tc##1{\textcolor[rgb]{0.74,0.48,0.00}{##1}}}
\@namedef{PY@tok@k}{\let\PY@bf=\textbf\def\PY@tc##1{\textcolor[rgb]{0.00,0.50,0.00}{##1}}}
\@namedef{PY@tok@kp}{\def\PY@tc##1{\textcolor[rgb]{0.00,0.50,0.00}{##1}}}
\@namedef{PY@tok@kt}{\def\PY@tc##1{\textcolor[rgb]{0.69,0.00,0.25}{##1}}}
\@namedef{PY@tok@o}{\def\PY@tc##1{\textcolor[rgb]{0.40,0.40,0.40}{##1}}}
\@namedef{PY@tok@ow}{\let\PY@bf=\textbf\def\PY@tc##1{\textcolor[rgb]{0.67,0.13,1.00}{##1}}}
\@namedef{PY@tok@nb}{\def\PY@tc##1{\textcolor[rgb]{0.00,0.50,0.00}{##1}}}
\@namedef{PY@tok@nf}{\def\PY@tc##1{\textcolor[rgb]{0.00,0.00,1.00}{##1}}}
\@namedef{PY@tok@nc}{\let\PY@bf=\textbf\def\PY@tc##1{\textcolor[rgb]{0.00,0.00,1.00}{##1}}}
\@namedef{PY@tok@nn}{\let\PY@bf=\textbf\def\PY@tc##1{\textcolor[rgb]{0.00,0.00,1.00}{##1}}}
\@namedef{PY@tok@ne}{\let\PY@bf=\textbf\def\PY@tc##1{\textcolor[rgb]{0.82,0.25,0.23}{##1}}}
\@namedef{PY@tok@nv}{\def\PY@tc##1{\textcolor[rgb]{0.10,0.09,0.49}{##1}}}
\@namedef{PY@tok@no}{\def\PY@tc##1{\textcolor[rgb]{0.53,0.00,0.00}{##1}}}
\@namedef{PY@tok@nl}{\def\PY@tc##1{\textcolor[rgb]{0.63,0.63,0.00}{##1}}}
\@namedef{PY@tok@ni}{\let\PY@bf=\textbf\def\PY@tc##1{\textcolor[rgb]{0.60,0.60,0.60}{##1}}}
\@namedef{PY@tok@na}{\def\PY@tc##1{\textcolor[rgb]{0.49,0.56,0.16}{##1}}}
\@namedef{PY@tok@nt}{\let\PY@bf=\textbf\def\PY@tc##1{\textcolor[rgb]{0.00,0.50,0.00}{##1}}}
\@namedef{PY@tok@nd}{\def\PY@tc##1{\textcolor[rgb]{0.67,0.13,1.00}{##1}}}
\@namedef{PY@tok@s}{\def\PY@tc##1{\textcolor[rgb]{0.73,0.13,0.13}{##1}}}
\@namedef{PY@tok@sd}{\let\PY@it=\textit\def\PY@tc##1{\textcolor[rgb]{0.73,0.13,0.13}{##1}}}
\@namedef{PY@tok@si}{\let\PY@bf=\textbf\def\PY@tc##1{\textcolor[rgb]{0.73,0.40,0.53}{##1}}}
\@namedef{PY@tok@se}{\let\PY@bf=\textbf\def\PY@tc##1{\textcolor[rgb]{0.73,0.40,0.13}{##1}}}
\@namedef{PY@tok@sr}{\def\PY@tc##1{\textcolor[rgb]{0.73,0.40,0.53}{##1}}}
\@namedef{PY@tok@ss}{\def\PY@tc##1{\textcolor[rgb]{0.10,0.09,0.49}{##1}}}
\@namedef{PY@tok@sx}{\def\PY@tc##1{\textcolor[rgb]{0.00,0.50,0.00}{##1}}}
\@namedef{PY@tok@m}{\def\PY@tc##1{\textcolor[rgb]{0.40,0.40,0.40}{##1}}}
\@namedef{PY@tok@gh}{\let\PY@bf=\textbf\def\PY@tc##1{\textcolor[rgb]{0.00,0.00,0.50}{##1}}}
\@namedef{PY@tok@gu}{\let\PY@bf=\textbf\def\PY@tc##1{\textcolor[rgb]{0.50,0.00,0.50}{##1}}}
\@namedef{PY@tok@gd}{\def\PY@tc##1{\textcolor[rgb]{0.63,0.00,0.00}{##1}}}
\@namedef{PY@tok@gi}{\def\PY@tc##1{\textcolor[rgb]{0.00,0.63,0.00}{##1}}}
\@namedef{PY@tok@gr}{\def\PY@tc##1{\textcolor[rgb]{1.00,0.00,0.00}{##1}}}
\@namedef{PY@tok@ge}{\let\PY@it=\textit}
\@namedef{PY@tok@gs}{\let\PY@bf=\textbf}
\@namedef{PY@tok@gp}{\let\PY@bf=\textbf\def\PY@tc##1{\textcolor[rgb]{0.00,0.00,0.50}{##1}}}
\@namedef{PY@tok@go}{\def\PY@tc##1{\textcolor[rgb]{0.53,0.53,0.53}{##1}}}
\@namedef{PY@tok@gt}{\def\PY@tc##1{\textcolor[rgb]{0.00,0.27,0.87}{##1}}}
\@namedef{PY@tok@err}{\def\PY@bc##1{{\setlength{\fboxsep}{\string -\fboxrule}\fcolorbox[rgb]{1.00,0.00,0.00}{1,1,1}{\strut ##1}}}}
\@namedef{PY@tok@kc}{\let\PY@bf=\textbf\def\PY@tc##1{\textcolor[rgb]{0.00,0.50,0.00}{##1}}}
\@namedef{PY@tok@kd}{\let\PY@bf=\textbf\def\PY@tc##1{\textcolor[rgb]{0.00,0.50,0.00}{##1}}}
\@namedef{PY@tok@kn}{\let\PY@bf=\textbf\def\PY@tc##1{\textcolor[rgb]{0.00,0.50,0.00}{##1}}}
\@namedef{PY@tok@kr}{\let\PY@bf=\textbf\def\PY@tc##1{\textcolor[rgb]{0.00,0.50,0.00}{##1}}}
\@namedef{PY@tok@bp}{\def\PY@tc##1{\textcolor[rgb]{0.00,0.50,0.00}{##1}}}
\@namedef{PY@tok@fm}{\def\PY@tc##1{\textcolor[rgb]{0.00,0.00,1.00}{##1}}}
\@namedef{PY@tok@vc}{\def\PY@tc##1{\textcolor[rgb]{0.10,0.09,0.49}{##1}}}
\@namedef{PY@tok@vg}{\def\PY@tc##1{\textcolor[rgb]{0.10,0.09,0.49}{##1}}}
\@namedef{PY@tok@vi}{\def\PY@tc##1{\textcolor[rgb]{0.10,0.09,0.49}{##1}}}
\@namedef{PY@tok@vm}{\def\PY@tc##1{\textcolor[rgb]{0.10,0.09,0.49}{##1}}}
\@namedef{PY@tok@sa}{\def\PY@tc##1{\textcolor[rgb]{0.73,0.13,0.13}{##1}}}
\@namedef{PY@tok@sb}{\def\PY@tc##1{\textcolor[rgb]{0.73,0.13,0.13}{##1}}}
\@namedef{PY@tok@sc}{\def\PY@tc##1{\textcolor[rgb]{0.73,0.13,0.13}{##1}}}
\@namedef{PY@tok@dl}{\def\PY@tc##1{\textcolor[rgb]{0.73,0.13,0.13}{##1}}}
\@namedef{PY@tok@s2}{\def\PY@tc##1{\textcolor[rgb]{0.73,0.13,0.13}{##1}}}
\@namedef{PY@tok@sh}{\def\PY@tc##1{\textcolor[rgb]{0.73,0.13,0.13}{##1}}}
\@namedef{PY@tok@s1}{\def\PY@tc##1{\textcolor[rgb]{0.73,0.13,0.13}{##1}}}
\@namedef{PY@tok@mb}{\def\PY@tc##1{\textcolor[rgb]{0.40,0.40,0.40}{##1}}}
\@namedef{PY@tok@mf}{\def\PY@tc##1{\textcolor[rgb]{0.40,0.40,0.40}{##1}}}
\@namedef{PY@tok@mh}{\def\PY@tc##1{\textcolor[rgb]{0.40,0.40,0.40}{##1}}}
\@namedef{PY@tok@mi}{\def\PY@tc##1{\textcolor[rgb]{0.40,0.40,0.40}{##1}}}
\@namedef{PY@tok@il}{\def\PY@tc##1{\textcolor[rgb]{0.40,0.40,0.40}{##1}}}
\@namedef{PY@tok@mo}{\def\PY@tc##1{\textcolor[rgb]{0.40,0.40,0.40}{##1}}}
\@namedef{PY@tok@ch}{\let\PY@it=\textit\def\PY@tc##1{\textcolor[rgb]{0.25,0.50,0.50}{##1}}}
\@namedef{PY@tok@cm}{\let\PY@it=\textit\def\PY@tc##1{\textcolor[rgb]{0.25,0.50,0.50}{##1}}}
\@namedef{PY@tok@cpf}{\let\PY@it=\textit\def\PY@tc##1{\textcolor[rgb]{0.25,0.50,0.50}{##1}}}
\@namedef{PY@tok@c1}{\let\PY@it=\textit\def\PY@tc##1{\textcolor[rgb]{0.25,0.50,0.50}{##1}}}
\@namedef{PY@tok@cs}{\let\PY@it=\textit\def\PY@tc##1{\textcolor[rgb]{0.25,0.50,0.50}{##1}}}

\def\PYZbs{\char`\\}
\def\PYZus{\char`\_}
\def\PYZob{\char`\{}
\def\PYZcb{\char`\}}
\def\PYZca{\char`\^}
\def\PYZam{\char`\&}
\def\PYZlt{\char`\<}
\def\PYZgt{\char`\>}
\def\PYZsh{\char`\#}
\def\PYZpc{\char`\%}
\def\PYZdl{\char`\$}
\def\PYZhy{\char`\-}
\def\PYZsq{\char`\'}
\def\PYZdq{\char`\"}
\def\PYZti{\char`\~}
% for compatibility with earlier versions
\def\PYZat{@}
\def\PYZlb{[}
\def\PYZrb{]}
\makeatother


    % For linebreaks inside Verbatim environment from package fancyvrb. 
    \makeatletter
        \newbox\Wrappedcontinuationbox 
        \newbox\Wrappedvisiblespacebox 
        \newcommand*\Wrappedvisiblespace {\textcolor{red}{\textvisiblespace}} 
        \newcommand*\Wrappedcontinuationsymbol {\textcolor{red}{\llap{\tiny$\m@th\hookrightarrow$}}} 
        \newcommand*\Wrappedcontinuationindent {3ex } 
        \newcommand*\Wrappedafterbreak {\kern\Wrappedcontinuationindent\copy\Wrappedcontinuationbox} 
        % Take advantage of the already applied Pygments mark-up to insert 
        % potential linebreaks for TeX processing. 
        %        {, <, #, %, $, ' and ": go to next line. 
        %        _, }, ^, &, >, - and ~: stay at end of broken line. 
        % Use of \textquotesingle for straight quote. 
        \newcommand*\Wrappedbreaksatspecials {% 
            \def\PYGZus{\discretionary{\char`\_}{\Wrappedafterbreak}{\char`\_}}% 
            \def\PYGZob{\discretionary{}{\Wrappedafterbreak\char`\{}{\char`\{}}% 
            \def\PYGZcb{\discretionary{\char`\}}{\Wrappedafterbreak}{\char`\}}}% 
            \def\PYGZca{\discretionary{\char`\^}{\Wrappedafterbreak}{\char`\^}}% 
            \def\PYGZam{\discretionary{\char`\&}{\Wrappedafterbreak}{\char`\&}}% 
            \def\PYGZlt{\discretionary{}{\Wrappedafterbreak\char`\<}{\char`\<}}% 
            \def\PYGZgt{\discretionary{\char`\>}{\Wrappedafterbreak}{\char`\>}}% 
            \def\PYGZsh{\discretionary{}{\Wrappedafterbreak\char`\#}{\char`\#}}% 
            \def\PYGZpc{\discretionary{}{\Wrappedafterbreak\char`\%}{\char`\%}}% 
            \def\PYGZdl{\discretionary{}{\Wrappedafterbreak\char`\$}{\char`\$}}% 
            \def\PYGZhy{\discretionary{\char`\-}{\Wrappedafterbreak}{\char`\-}}% 
            \def\PYGZsq{\discretionary{}{\Wrappedafterbreak\textquotesingle}{\textquotesingle}}% 
            \def\PYGZdq{\discretionary{}{\Wrappedafterbreak\char`\"}{\char`\"}}% 
            \def\PYGZti{\discretionary{\char`\~}{\Wrappedafterbreak}{\char`\~}}% 
        } 
        % Some characters . , ; ? ! / are not pygmentized. 
        % This macro makes them "active" and they will insert potential linebreaks 
        \newcommand*\Wrappedbreaksatpunct {% 
            \lccode`\~`\.\lowercase{\def~}{\discretionary{\hbox{\char`\.}}{\Wrappedafterbreak}{\hbox{\char`\.}}}% 
            \lccode`\~`\,\lowercase{\def~}{\discretionary{\hbox{\char`\,}}{\Wrappedafterbreak}{\hbox{\char`\,}}}% 
            \lccode`\~`\;\lowercase{\def~}{\discretionary{\hbox{\char`\;}}{\Wrappedafterbreak}{\hbox{\char`\;}}}% 
            \lccode`\~`\:\lowercase{\def~}{\discretionary{\hbox{\char`\:}}{\Wrappedafterbreak}{\hbox{\char`\:}}}% 
            \lccode`\~`\?\lowercase{\def~}{\discretionary{\hbox{\char`\?}}{\Wrappedafterbreak}{\hbox{\char`\?}}}% 
            \lccode`\~`\!\lowercase{\def~}{\discretionary{\hbox{\char`\!}}{\Wrappedafterbreak}{\hbox{\char`\!}}}% 
            \lccode`\~`\/\lowercase{\def~}{\discretionary{\hbox{\char`\/}}{\Wrappedafterbreak}{\hbox{\char`\/}}}% 
            \catcode`\.\active
            \catcode`\,\active 
            \catcode`\;\active
            \catcode`\:\active
            \catcode`\?\active
            \catcode`\!\active
            \catcode`\/\active 
            \lccode`\~`\~ 	
        }
    \makeatother

    \let\OriginalVerbatim=\Verbatim
    \makeatletter
    \renewcommand{\Verbatim}[1][1]{%
        %\parskip\z@skip
        \sbox\Wrappedcontinuationbox {\Wrappedcontinuationsymbol}%
        \sbox\Wrappedvisiblespacebox {\FV@SetupFont\Wrappedvisiblespace}%
        \def\FancyVerbFormatLine ##1{\hsize\linewidth
            \vtop{\raggedright\hyphenpenalty\z@\exhyphenpenalty\z@
                \doublehyphendemerits\z@\finalhyphendemerits\z@
                \strut ##1\strut}%
        }%
        % If the linebreak is at a space, the latter will be displayed as visible
        % space at end of first line, and a continuation symbol starts next line.
        % Stretch/shrink are however usually zero for typewriter font.
        \def\FV@Space {%
            \nobreak\hskip\z@ plus\fontdimen3\font minus\fontdimen4\font
            \discretionary{\copy\Wrappedvisiblespacebox}{\Wrappedafterbreak}
            {\kern\fontdimen2\font}%
        }%
        
        % Allow breaks at special characters using \PYG... macros.
        \Wrappedbreaksatspecials
        % Breaks at punctuation characters . , ; ? ! and / need catcode=\active 	
        \OriginalVerbatim[#1,codes*=\Wrappedbreaksatpunct]%
    }
    \makeatother

    % Exact colors from NB
    \definecolor{incolor}{HTML}{303F9F}
    \definecolor{outcolor}{HTML}{D84315}
    \definecolor{cellborder}{HTML}{CFCFCF}
    \definecolor{cellbackground}{HTML}{F7F7F7}
    
    % prompt
    \makeatletter
    \newcommand{\boxspacing}{\kern\kvtcb@left@rule\kern\kvtcb@boxsep}
    \makeatother
    \newcommand{\prompt}[4]{
        {\ttfamily\llap{{\color{#2}[#3]:\hspace{3pt}#4}}\vspace{-\baselineskip}}
    }
    

    
    % Prevent overflowing lines due to hard-to-break entities
    \sloppy 
    % Setup hyperref package
    \hypersetup{
      breaklinks=true,  % so long urls are correctly broken across lines
      colorlinks=true,
      urlcolor=urlcolor,
      linkcolor=linkcolor,
      citecolor=citecolor,
      }
    % Slightly bigger margins than the latex defaults
    
    \geometry{verbose,tmargin=1in,bmargin=1in,lmargin=1in,rmargin=1in}
    
    

\begin{document}
	 \begin{titlepage}
		\begin{center}
			\fontsize{12pt}{12pt}\selectfont{
				\textbf{МИНИСТЕРСТВО ОБРАЗОВАНИЯ РЕСПУБЛИКИ БЕЛАРУСЬ \\
					БЕЛОРУССКИЙ ГОСУДАРСТВЕННЫЙ УНИВЕРСИТЕТ} \\
				\vspace{0.5cm}
				\textbf{ФАКУЛЬТЕТ ПРИКЛАДНОЙ МАТЕМАТИКИ И ИНФОРМАТИКИ \\
					Кафедра математического моделирования и анализа данных}
			}
		\end{center}
		\vspace*{\fill}
		\fontsize{14pt}{14pt}\selectfont{
			\begin{center}
				АВСЯННИК \\
				Елизавета Дмитриевна \\
				\vspace{1cm}
				\textbf{Лабораторная работа №3} \\
				\vspace{1cm}
				Имитационное и статистическоге моделирование
			\end{center}
			\vspace{1cm}
			\begin{flushright}
				\begin{tabular}{@{}l@{}}
					Проверил: \\
					В.П. Кирлица
				\end{tabular}
			\end{flushright}
		}
		\vspace*{\fill}
		\begin{center}
			\fontsize{14pt}{14pt}\selectfont
			Минск, 2021
		\end{center}
	\end{titlepage}
	\pagenumbering{arabic}
	\setcounter{page}{2}
    

    
    

    
    \hypertarget{ux43bux430ux431ux43eux440ux430ux442ux43eux440ux43dux430ux44f-ux440ux430ux431ux43eux442ux430-3}{%
\subsubsection{Лабораторная работа
3}\label{ux43bux430ux431ux43eux440ux430ux442ux43eux440ux43dux430ux44f-ux440ux430ux431ux43eux442ux430-3}}

    Используя метод Монте-Карло вычислить интеграл:

\(I = \int_0^{\pi/2}cosxdx\).

Сравнить полученную оценку с оценкой, полученной по методу выделения
главной части (\(h(x) = 1 - \frac{x^2}2\)). Сравнить дисперсии этих
оценок.

    \begin{tcolorbox}[breakable, size=fbox, boxrule=1pt, pad at break*=1mm,colback=cellbackground, colframe=cellborder]
\prompt{In}{incolor}{87}{\boxspacing}
\begin{Verbatim}[commandchars=\\\{\}]
\PY{k+kn}{import} \PY{n+nn}{numpy} \PY{k}{as} \PY{n+nn}{np}
\PY{k+kn}{import} \PY{n+nn}{pandas} \PY{k}{as} \PY{n+nn}{pd}
\PY{k+kn}{from} \PY{n+nn}{math} \PY{k+kn}{import} \PY{n}{floor}\PY{p}{,} \PY{n}{pi}\PY{p}{,} \PY{n}{cos}
\PY{k+kn}{import} \PY{n+nn}{time}
\PY{k+kn}{from} \PY{n+nn}{scipy} \PY{k+kn}{import} \PY{n}{integrate}

\PY{k+kn}{import} \PY{n+nn}{typing} \PY{k}{as} \PY{n+nn}{tp}

\PY{k+kn}{from} \PY{n+nn}{tabulate} \PY{k+kn}{import} \PY{n}{tabulate}
\end{Verbatim}
\end{tcolorbox}

    \hypertarget{ux43cux443ux43bux44cux442ux438ux43fux43bux438ux43aux430ux442ux438ux432ux43dux44bux439-ux43aux43eux43dux433ux440ux443ux44dux43dux442ux43dux44bux439-ux43cux435ux442ux43eux434-ux43cux43eux434ux435ux43bux438ux440ux43eux432ux430ux43dux438ux44f-ux431ux441ux432}{%
\subparagraph{Мультипликативный конгруэнтный метод моделирования
БСВ}\label{ux43cux443ux43bux44cux442ux438ux43fux43bux438ux43aux430ux442ux438ux432ux43dux44bux439-ux43aux43eux43dux433ux440ux443ux44dux43dux442ux43dux44bux439-ux43cux435ux442ux43eux434-ux43cux43eux434ux435ux43bux438ux440ux43eux432ux430ux43dux438ux44f-ux431ux441ux432}}

    \begin{tcolorbox}[breakable, size=fbox, boxrule=1pt, pad at break*=1mm,colback=cellbackground, colframe=cellborder]
\prompt{In}{incolor}{88}{\boxspacing}
\begin{Verbatim}[commandchars=\\\{\}]
\PY{k}{def} \PY{n+nf}{generate\PYZus{}brv\PYZus{}congruential\PYZus{}sample}\PY{p}{(}\PY{n}{M}\PY{p}{:} \PY{n+nb}{int}\PY{o}{=}\PY{l+m+mi}{2}\PY{o}{*}\PY{o}{*}\PY{l+m+mi}{31}\PY{p}{,} \PY{n}{alpha\PYZus{}star0}\PY{p}{:} \PY{n+nb}{int}\PY{o}{=}\PY{l+m+mi}{65539}\PY{p}{,} \PY{n}{beta} \PY{p}{:}\PY{n+nb}{int}\PY{o}{=}\PY{l+m+mi}{65539}\PY{p}{)} \PY{o}{\PYZhy{}}\PY{o}{\PYZgt{}} \PY{n+nb}{float}\PY{p}{:}
    \PY{n}{alpha\PYZus{}t} \PY{o}{=} \PY{n}{alpha\PYZus{}star0}
    \PY{k}{while} \PY{k+kc}{True}\PY{p}{:}
        \PY{n}{alpha\PYZus{}t} \PY{o}{=} \PY{p}{(}\PY{n}{alpha\PYZus{}t} \PY{o}{*} \PY{n}{beta}\PY{p}{)} \PY{o}{\PYZpc{}} \PY{n}{M}
        \PY{k}{yield} \PY{n}{alpha\PYZus{}t} \PY{o}{/} \PY{n}{M}
        
\PY{k}{def} \PY{n+nf}{generate\PYZus{}brv\PYZus{}congruential}\PY{p}{(}\PY{n}{n}\PY{p}{:} \PY{n+nb}{int}\PY{o}{=}\PY{l+m+mi}{100}\PY{p}{,} \PY{n}{M}\PY{p}{:} \PY{n+nb}{int}\PY{o}{=}\PY{l+m+mi}{2}\PY{o}{*}\PY{o}{*}\PY{l+m+mi}{31}\PY{p}{,} \PY{n}{alpha\PYZus{}star0}\PY{p}{:} \PY{n+nb}{int}\PY{o}{=}\PY{l+m+mi}{65539}\PY{p}{,} \PY{n}{beta} \PY{p}{:}\PY{n+nb}{int}\PY{o}{=}\PY{l+m+mi}{65539}\PY{p}{)} \PY{o}{\PYZhy{}}\PY{o}{\PYZgt{}} \PY{n}{np}\PY{o}{.}\PY{n}{ndarray}\PY{p}{:} 
    \PY{n}{alpha} \PY{o}{=} \PY{n}{np}\PY{o}{.}\PY{n}{array}\PY{p}{(}\PY{p}{[}\PY{p}{]}\PY{p}{)}
    \PY{n}{generator} \PY{o}{=} \PY{n}{generate\PYZus{}brv\PYZus{}congruential\PYZus{}sample}\PY{p}{(}\PY{n}{M}\PY{o}{=}\PY{n}{M}\PY{p}{,} \PY{n}{alpha\PYZus{}star0}\PY{o}{=}\PY{n}{alpha\PYZus{}star0}\PY{p}{,} \PY{n}{beta}\PY{o}{=}\PY{n}{beta}\PY{p}{)}
    \PY{k}{for} \PY{n}{i} \PY{o+ow}{in} \PY{n+nb}{range}\PY{p}{(}\PY{n}{n}\PY{p}{)}\PY{p}{:} 
        \PY{n}{alpha} \PY{o}{=} \PY{n}{np}\PY{o}{.}\PY{n}{append}\PY{p}{(}\PY{n}{alpha}\PY{p}{,} \PY{n+nb}{next}\PY{p}{(}\PY{n}{generator}\PY{p}{)}\PY{p}{)}
    \PY{k}{return} \PY{n}{alpha}
\end{Verbatim}
\end{tcolorbox}

    \hypertarget{ux43cux435ux442ux43eux434-ux43cux430ux43aux43bux430ux440ux435ux43dux430-ux43cux430ux440ux441ux430ux43bux44cux438-ux43cux43eux434ux435ux43bux438ux440ux43eux432ux430ux43dux438ux44f-ux431ux441ux432-ux434ux43bux44f-ux43cux43eux434ux435ux43bux438ux440ux43eux432ux430ux43dux438ux44f-ux440ux430ux432ux43dux43eux43cux435ux440ux43dux43eux433ux43e-ux440ux430ux441ux43fux440ux435ux434ux435ux43bux435ux43dux438ux44f-ux43dux430-ux43eux442ux440ux435ux437ux43aux435}{%
\subparagraph{Метод Макларена-Марсальи моделирования БСВ для
моделирования равномерного распределения на
отрезке}\label{ux43cux435ux442ux43eux434-ux43cux430ux43aux43bux430ux440ux435ux43dux430-ux43cux430ux440ux441ux430ux43bux44cux438-ux43cux43eux434ux435ux43bux438ux440ux43eux432ux430ux43dux438ux44f-ux431ux441ux432-ux434ux43bux44f-ux43cux43eux434ux435ux43bux438ux440ux43eux432ux430ux43dux438ux44f-ux440ux430ux432ux43dux43eux43cux435ux440ux43dux43eux433ux43e-ux440ux430ux441ux43fux440ux435ux434ux435ux43bux435ux43dux438ux44f-ux43dux430-ux43eux442ux440ux435ux437ux43aux435}}

    \begin{tcolorbox}[breakable, size=fbox, boxrule=1pt, pad at break*=1mm,colback=cellbackground, colframe=cellborder]
\prompt{In}{incolor}{89}{\boxspacing}
\begin{Verbatim}[commandchars=\\\{\}]
\PY{k}{def} \PY{n+nf}{generate\PYZus{}brv\PYZus{}mm\PYZus{}sample}\PY{p}{(}\PY{n}{k}\PY{p}{:} \PY{n+nb}{int}\PY{o}{=}\PY{l+m+mi}{128}\PY{p}{,} \PY{n}{a}\PY{p}{:}\PY{n+nb}{float}\PY{o}{=}\PY{l+m+mi}{0}\PY{p}{,} \PY{n}{b}\PY{p}{:}\PY{n+nb}{float}\PY{o}{=}\PY{l+m+mi}{1}\PY{p}{)} \PY{o}{\PYZhy{}}\PY{o}{\PYZgt{}} \PY{n+nb}{float}\PY{p}{:}
    \PY{n}{v} \PY{o}{=} \PY{n}{np}\PY{o}{.}\PY{n}{array}\PY{p}{(}\PY{p}{[}\PY{p}{]}\PY{p}{)}
    \PY{n}{brv\PYZus{}cong\PYZus{}b} \PY{o}{=} \PY{n}{generate\PYZus{}brv\PYZus{}congruential\PYZus{}sample}\PY{p}{(}\PY{p}{)}
    \PY{n}{t} \PY{o}{=} \PY{n}{time}\PY{o}{.}\PY{n}{perf\PYZus{}counter}\PY{p}{(}\PY{p}{)}
    \PY{n}{alpha\PYZus{}star0} \PY{o}{=} \PY{n}{beta} \PY{o}{=}  \PY{n+nb}{int}\PY{p}{(}\PY{l+m+mi}{10}\PY{o}{*}\PY{o}{*}\PY{l+m+mi}{9}\PY{o}{*}\PY{n+nb}{float}\PY{p}{(}\PY{p}{(}\PY{n}{t}\PY{o}{\PYZhy{}}\PY{n+nb}{int}\PY{p}{(}\PY{n}{t}\PY{p}{)}\PY{p}{)}\PY{p}{)}\PY{p}{)}
    \PY{n}{brv\PYZus{}cong\PYZus{}c} \PY{o}{=} \PY{n}{generate\PYZus{}brv\PYZus{}congruential\PYZus{}sample}\PY{p}{(}\PY{n}{alpha\PYZus{}star0}\PY{o}{=}\PY{n}{alpha\PYZus{}star0}\PY{p}{,} \PY{n}{beta}\PY{o}{=}\PY{n}{beta}\PY{p}{)}
    \PY{k}{for} \PY{n}{i} \PY{o+ow}{in} \PY{n+nb}{range}\PY{p}{(}\PY{n}{k}\PY{p}{)}\PY{p}{:}
            \PY{n}{v} \PY{o}{=} \PY{n}{np}\PY{o}{.}\PY{n}{append}\PY{p}{(}\PY{n}{v}\PY{p}{,} \PY{n+nb}{next}\PY{p}{(}\PY{n}{brv\PYZus{}cong\PYZus{}b}\PY{p}{)}\PY{p}{)}
    \PY{k}{while} \PY{k+kc}{True}\PY{p}{:}
        \PY{n}{index} \PY{o}{=} \PY{n}{floor}\PY{p}{(}\PY{n+nb}{next}\PY{p}{(}\PY{n}{brv\PYZus{}cong\PYZus{}c}\PY{p}{)} \PY{o}{*} \PY{n}{k}\PY{p}{)}
        \PY{n}{alpha\PYZus{}t} \PY{o}{=} \PY{n}{v}\PY{p}{[}\PY{n}{index}\PY{p}{]}
        \PY{n}{v}\PY{p}{[}\PY{n}{index}\PY{p}{]} \PY{o}{=} \PY{n+nb}{next}\PY{p}{(}\PY{n}{brv\PYZus{}cong\PYZus{}b}\PY{p}{)}
        \PY{k}{yield} \PY{n}{alpha\PYZus{}t} \PY{o}{*} \PY{p}{(}\PY{n}{b} \PY{o}{\PYZhy{}} \PY{n}{a}\PY{p}{)} \PY{o}{+} \PY{n}{a}
        
\PY{k}{def} \PY{n+nf}{generate\PYZus{}brv\PYZus{}mm}\PY{p}{(}\PY{n}{n}\PY{p}{:} \PY{n+nb}{int}\PY{o}{=}\PY{l+m+mi}{100}\PY{p}{,} \PY{n}{k}\PY{p}{:} \PY{n+nb}{int}\PY{o}{=}\PY{l+m+mi}{128}\PY{p}{,}  \PY{n}{a}\PY{p}{:}\PY{n+nb}{float}\PY{o}{=}\PY{l+m+mi}{0}\PY{p}{,} \PY{n}{b}\PY{p}{:}\PY{n+nb}{float}\PY{o}{=}\PY{l+m+mi}{1}\PY{p}{)} \PY{o}{\PYZhy{}}\PY{o}{\PYZgt{}} \PY{n}{np}\PY{o}{.}\PY{n}{ndarray}\PY{p}{:} 
    \PY{n}{alpha} \PY{o}{=} \PY{n}{np}\PY{o}{.}\PY{n}{array}\PY{p}{(}\PY{p}{[}\PY{p}{]}\PY{p}{)}
    \PY{n}{generator} \PY{o}{=} \PY{n}{generate\PYZus{}brv\PYZus{}mm\PYZus{}sample}\PY{p}{(}\PY{n}{k}\PY{o}{=}\PY{n}{k}\PY{p}{,} \PY{n}{a}\PY{o}{=}\PY{n}{a}\PY{p}{,} \PY{n}{b}\PY{o}{=}\PY{n}{b}\PY{p}{)}
    \PY{k}{for} \PY{n}{i} \PY{o+ow}{in} \PY{n+nb}{range}\PY{p}{(}\PY{n}{n}\PY{p}{)}\PY{p}{:}
        \PY{n}{alpha} \PY{o}{=} \PY{n}{np}\PY{o}{.}\PY{n}{append}\PY{p}{(}\PY{n}{alpha}\PY{p}{,} \PY{n+nb}{next}\PY{p}{(}\PY{n}{generator}\PY{p}{)}\PY{p}{)}
    \PY{k}{return} \PY{n}{alpha}
\end{Verbatim}
\end{tcolorbox}

    \hypertarget{ux43eux441ux43dux43eux432ux43dux43eux439-ux446ux438ux43aux43b-ux43fux440ux43eux433ux440ux430ux43cux43cux44b}{%
\subsubsection{Основной цикл
программы}\label{ux43eux441ux43dux43eux432ux43dux43eux439-ux446ux438ux43aux43b-ux43fux440ux43eux433ux440ux430ux43cux43cux44b}}

    \textbf{\emph{Метод Монте-Карло:}}

В качестве \(\xi\) была выбрана СВ равномерно распределенная на отрезке
\([0, pi/2]\).

Таким образом:
\(p_{\xi}(x) = 2 / \pi > 0, x \in [x_0, x_1] \int_{x_0}^{x_1}p_{\xi}(x)dx = 1\),
где \(x_0 = 0, x_1 = \pi / 2\)

\(\eta = g(\xi) = \frac{cos\xi}{p_{\xi}(x)} = \frac{\pi cos\xi}{2} \)

Можно показать, что \(M\{\eta\} = I, D\{\eta\}< \infty\). Поэтому в
качестве приближенного значения интеграла можно использовать
статистическую оценку \(I_{n}\), построенную в выборке из \(n\)
независимых случайных величин \(\eta_{1}, \eta_{2}... \eta_{n}\):

\(I_{n} = \frac{1}{n} \sum_{i=0}^{n}\eta_i = \frac{1}{n} \sum_{i=0}^{n}\frac{g(\xi_i)}{p_\xi(\xi_i)}\)

    \textbf{\emph{Метод выделения главной части:}}

\(I = \int_0^{\pi/2}cosxdx = \int_0^{\pi/2}g(x)p_{\xi}(x)dx \int_0^{\pi/2}\frac{\pi}{2}cosx\frac{2}{\pi}dx\)

\(h(x) = \frac{\pi}{2}(1 - \frac{x^2}2)\)

\(a = \int_0^{\pi/2}h(x)\frac{2}{\pi}dx = \int_0^{\pi/2}(1 - \frac{x^2}2)dx = -\frac{1}{48}\pi(\pi^2 - 24)\)

\(\eta_1 = g_1(\xi)= a + \frac{cos(\xi)}{p_{\xi}(x)} - \frac{h(\xi)}{p_{\xi}(x)} = a + cos(\xi)\frac{\pi}{2} - (1 - \frac{\xi^2}2)\frac{\pi}{2}\)

    \begin{tcolorbox}[breakable, size=fbox, boxrule=1pt, pad at break*=1mm,colback=cellbackground, colframe=cellborder]
\prompt{In}{incolor}{90}{\boxspacing}
\begin{Verbatim}[commandchars=\\\{\}]
\PY{n}{x\PYZus{}0} \PY{o}{=} \PY{l+m+mi}{0}
\PY{n}{x\PYZus{}1} \PY{o}{=} \PY{n}{pi} \PY{o}{/} \PY{l+m+mi}{2}
\PY{n}{p} \PY{o}{=} \PY{l+m+mi}{2} \PY{o}{/} \PY{n}{pi}

\PY{n}{h} \PY{o}{=} \PY{k}{lambda} \PY{n}{x}\PY{p}{:} \PY{p}{(}\PY{l+m+mi}{1} \PY{o}{\PYZhy{}} \PY{n}{x}\PY{o}{*}\PY{o}{*}\PY{l+m+mi}{2} \PY{o}{/} \PY{l+m+mi}{2}\PY{p}{)}
\PY{n}{a} \PY{o}{=} \PY{n}{integrate}\PY{o}{.}\PY{n}{quad}\PY{p}{(}\PY{n}{h}\PY{p}{,} \PY{n}{x\PYZus{}0}\PY{p}{,} \PY{n}{x\PYZus{}1}\PY{p}{)}\PY{p}{[}\PY{l+m+mi}{0}\PY{p}{]}

\PY{k}{def} \PY{n+nf}{cos\PYZus{}main\PYZus{}component}\PY{p}{(}\PY{n}{x}\PY{p}{:}\PY{n+nb}{float}\PY{o}{=}\PY{l+m+mi}{0}\PY{p}{)} \PY{o}{\PYZhy{}}\PY{o}{\PYZgt{}} \PY{n+nb}{float}\PY{p}{:}
    \PY{k}{return} \PY{n}{a} \PY{o}{+} \PY{n}{cos}\PY{p}{(}\PY{n}{x}\PY{p}{)} \PY{o}{/} \PY{n}{p}  \PY{o}{\PYZhy{}} \PY{p}{(}\PY{l+m+mi}{1} \PY{o}{\PYZhy{}} \PY{n}{x}\PY{o}{*}\PY{o}{*}\PY{l+m+mi}{2} \PY{o}{/} \PY{l+m+mi}{2}\PY{p}{)} \PY{o}{/} \PY{n}{p}
    

\PY{n+nb}{print}\PY{p}{(}\PY{l+s+s2}{\PYZdq{}}\PY{l+s+s2}{Истинная величина: 1}\PY{l+s+s2}{\PYZdq{}}\PY{p}{)}

\PY{k}{for} \PY{n}{n} \PY{o+ow}{in} \PY{p}{[}\PY{l+m+mf}{1e3}\PY{p}{,} \PY{l+m+mf}{5e3}\PY{p}{,} \PY{l+m+mf}{1e4}\PY{p}{,} \PY{l+m+mf}{5e4}\PY{p}{,} \PY{l+m+mf}{1e5}\PY{p}{,} \PY{l+m+mf}{5e5}\PY{p}{,} \PY{l+m+mf}{1e6}\PY{p}{]}\PY{p}{:}
    \PY{n}{generator} \PY{o}{=} \PY{n}{generate\PYZus{}brv\PYZus{}mm\PYZus{}sample}\PY{p}{(}\PY{n}{a}\PY{o}{=}\PY{n}{x\PYZus{}0}\PY{p}{,} \PY{n}{b}\PY{o}{=}\PY{n}{x\PYZus{}1}\PY{p}{)}
    \PY{n}{result\PYZus{}monte\PYZus{}carlo} \PY{o}{=} \PY{p}{[}\PY{p}{]}
    \PY{n}{result\PYZus{}main\PYZus{}component} \PY{o}{=} \PY{p}{[}\PY{p}{]}
    \PY{k}{for} \PY{n}{i} \PY{o+ow}{in} \PY{n+nb}{range}\PY{p}{(}\PY{n+nb}{int}\PY{p}{(}\PY{n}{n}\PY{p}{)}\PY{p}{)}\PY{p}{:}
        \PY{n}{rand} \PY{o}{=} \PY{n+nb}{next}\PY{p}{(}\PY{n}{generator}\PY{p}{)}
        \PY{n}{result\PYZus{}monte\PYZus{}carlo}\PY{o}{.}\PY{n}{append}\PY{p}{(}\PY{n}{cos}\PY{p}{(}\PY{n}{rand}\PY{p}{)} \PY{o}{/} \PY{n}{p}\PY{p}{)}
        \PY{n}{result\PYZus{}main\PYZus{}component}\PY{o}{.}\PY{n}{append}\PY{p}{(}\PY{n}{cos\PYZus{}main\PYZus{}component}\PY{p}{(}\PY{n}{rand}\PY{p}{)}\PY{p}{)}
    \PY{n}{result\PYZus{}monte\PYZus{}carlo} \PY{o}{=} \PY{n}{np}\PY{o}{.}\PY{n}{array}\PY{p}{(}\PY{n}{result\PYZus{}monte\PYZus{}carlo}\PY{p}{)}
    \PY{n}{result\PYZus{}main\PYZus{}component} \PY{o}{=} \PY{n}{np}\PY{o}{.}\PY{n}{array}\PY{p}{(}\PY{n}{result\PYZus{}main\PYZus{}component}\PY{p}{)}
    \PY{n+nb}{print}\PY{p}{(}\PY{l+s+sa}{f}\PY{l+s+s2}{\PYZdq{}}\PY{l+s+s2}{n = }\PY{l+s+si}{\PYZob{}}\PY{n+nb}{int}\PY{p}{(}\PY{n}{n}\PY{p}{)}\PY{l+s+si}{\PYZcb{}}\PY{l+s+s2}{\PYZdq{}}\PY{p}{)}
    \PY{n+nb}{print}\PY{p}{(}\PY{l+s+sa}{f}\PY{l+s+s2}{\PYZdq{}}\PY{l+s+s2}{Величина, используя метод Монте\PYZhy{}Карло: }\PY{l+s+si}{\PYZob{}}\PY{n}{result\PYZus{}monte\PYZus{}carlo}\PY{o}{.}\PY{n}{mean}\PY{p}{(}\PY{p}{)}\PY{l+s+si}{\PYZcb{}}\PY{l+s+s2}{, Дисперсия = }\PY{l+s+si}{\PYZob{}}\PY{n}{result\PYZus{}monte\PYZus{}carlo}\PY{o}{.}\PY{n}{var}\PY{p}{(}\PY{p}{)}\PY{l+s+si}{\PYZcb{}}\PY{l+s+s2}{\PYZdq{}}\PY{p}{)}
    \PY{n+nb}{print}\PY{p}{(}\PY{l+s+sa}{f}\PY{l+s+s2}{\PYZdq{}}\PY{l+s+s2}{Величина, используя метод выделения главной части: }\PY{l+s+si}{\PYZob{}}\PY{n}{result\PYZus{}main\PYZus{}component}\PY{o}{.}\PY{n}{mean}\PY{p}{(}\PY{p}{)}\PY{l+s+si}{\PYZcb{}}\PY{l+s+s2}{, Дисперсия = }\PY{l+s+si}{\PYZob{}}\PY{n}{result\PYZus{}main\PYZus{}component}\PY{o}{.}\PY{n}{var}\PY{p}{(}\PY{p}{)}\PY{l+s+si}{\PYZcb{}}\PY{l+s+s2}{\PYZdq{}}\PY{p}{)}
    
    
\end{Verbatim}
\end{tcolorbox}

    \begin{Verbatim}[commandchars=\\\{\}]
Истинная величина: 1
n = 1000
Величина, используя метод Монте-Карло: 0.9843287423637573, Дисперсия =
0.22941040163348658
Величина, используя метод выделения главной части: 1.0016535540498026, Дисперсия
= 0.009534640826746092
n = 5000
Величина, используя метод Монте-Карло: 1.0022757389426142, Дисперсия =
0.22936311887120583
Величина, используя метод выделения главной части: 0.9990209205504119, Дисперсия
= 0.009715987255176904
n = 10000
Величина, используя метод Монте-Карло: 1.0002487891845766, Дисперсия =
0.23243167728820102
Величина, используя метод выделения главной части: 0.999742549046459, Дисперсия
= 0.009692428618740302
n = 50000
Величина, используя метод Монте-Карло: 0.9965093450680339, Дисперсия =
0.23479704425321804
Величина, используя метод выделения главной части: 1.0007342830684043, Дисперсия
= 0.009901452269712327
n = 100000
Величина, используя метод Монте-Карло: 0.9967199297335406, Дисперсия =
0.234092183843278
Величина, используя метод выделения главной части: 1.0005721121178617, Дисперсия
= 0.009833764119653709
n = 500000
Величина, используя метод Монте-Карло: 1.0008996755188673, Дисперсия =
0.233713978701259
Величина, используя метод выделения главной части: 0.9998633312232682, Дисперсия
= 0.00977267225043554
n = 1000000
Величина, используя метод Монте-Карло: 1.000464464193042, Дисперсия =
0.23381803175895477
Величина, используя метод выделения главной части: 0.9999493008340916, Дисперсия
= 0.009786102825318416
    \end{Verbatim}

    \hypertarget{ux430ux43dux430ux43bux438ux442ux438ux447ux435ux441ux43aux430ux44f-ux43eux446ux435ux43dux43aux430-ux434ux438ux441ux43fux435ux440ux441ux438ux439}{%
\subsubsection{Аналитическая оценка
дисперсий}\label{ux430ux43dux430ux43bux438ux442ux438ux447ux435ux441ux43aux430ux44f-ux43eux446ux435ux43dux43aux430-ux434ux438ux441ux43fux435ux440ux441ux438ux439}}

    \textbf{\emph{Метод Монте-Карло}}

\(I = \int_0^{\pi/2}cosxdx = \int_0^{\pi/2}g(x)p_{\xi}(x)dx = 1\)

\(M\{\eta\} = I\)

\(D\{\eta\} = M\{\eta^2\}- M\{\eta\}^2 = \int_0^{\pi/2}\frac{\pi^2cos^2x}{4}\frac{2}{\pi}dx - 1 = \frac{\pi^2}{8} - 1 = 0.2337\)

    \begin{tcolorbox}[breakable, size=fbox, boxrule=1pt, pad at break*=1mm,colback=cellbackground, colframe=cellborder]
\prompt{In}{incolor}{91}{\boxspacing}
\begin{Verbatim}[commandchars=\\\{\}]
\PY{n}{pi}\PY{o}{*}\PY{o}{*}\PY{l+m+mi}{2} \PY{o}{/} \PY{l+m+mi}{8} \PY{o}{\PYZhy{}} \PY{l+m+mi}{1}
\end{Verbatim}
\end{tcolorbox}

            \begin{tcolorbox}[breakable, size=fbox, boxrule=.5pt, pad at break*=1mm, opacityfill=0]
\prompt{Out}{outcolor}{91}{\boxspacing}
\begin{Verbatim}[commandchars=\\\{\}]
0.23370055013616975
\end{Verbatim}
\end{tcolorbox}
        
    \textbf{\emph{Метод выделения главной части}}

\(M\{\eta_1\} = \int_0^{\pi/2}g_1(x)p_\xi(x)dx = \int_0^{\pi/2}(a + cos(x)\frac{\pi}{2} - (1 -\frac{x^2}2)\frac{\pi}{2})\frac{2}{\pi}dx = 1\)

\(D\{\eta_1\} = M\{\eta_1^2\}- M\{\eta_1\}^2 = \int_0^{\pi/2}(a + cos(x)\frac{\pi}{2} - (1 -\frac{x^2}2)\frac{\pi}{2})^2\frac{2}{\pi}dx - M\{\eta_1\}^2 = 0.0098\)

    \begin{tcolorbox}[breakable, size=fbox, boxrule=1pt, pad at break*=1mm,colback=cellbackground, colframe=cellborder]
\prompt{In}{incolor}{95}{\boxspacing}
\begin{Verbatim}[commandchars=\\\{\}]
\PY{c+c1}{\PYZsh{} Mат ожидание}
\PY{k+kn}{from} \PY{n+nn}{scipy} \PY{k+kn}{import} \PY{n}{integrate}
\PY{n}{m} \PY{o}{=} \PY{n}{integrate}\PY{o}{.}\PY{n}{quad}\PY{p}{(}\PY{n}{cos\PYZus{}main\PYZus{}component}\PY{p}{,} \PY{l+m+mi}{0}\PY{p}{,} \PY{n}{pi}\PY{o}{/}\PY{l+m+mi}{2}\PY{p}{)}\PY{p}{[}\PY{l+m+mi}{0}\PY{p}{]} \PY{o}{*} \PY{n}{p}
\PY{n}{m}
\end{Verbatim}
\end{tcolorbox}

            \begin{tcolorbox}[breakable, size=fbox, boxrule=.5pt, pad at break*=1mm, opacityfill=0]
\prompt{Out}{outcolor}{95}{\boxspacing}
\begin{Verbatim}[commandchars=\\\{\}]
1.0
\end{Verbatim}
\end{tcolorbox}
        
    \begin{tcolorbox}[breakable, size=fbox, boxrule=1pt, pad at break*=1mm,colback=cellbackground, colframe=cellborder]
\prompt{In}{incolor}{96}{\boxspacing}
\begin{Verbatim}[commandchars=\\\{\}]
\PY{c+c1}{\PYZsh{}Мат ожидание квадрата}
\PY{n}{m\PYZus{}2} \PY{o}{=} \PY{n}{integrate}\PY{o}{.}\PY{n}{quad}\PY{p}{(}\PY{k}{lambda} \PY{n}{x}\PY{p}{:} \PY{n}{cos\PYZus{}main\PYZus{}component}\PY{p}{(}\PY{n}{x}\PY{p}{)} \PY{o}{*}\PY{o}{*} \PY{l+m+mi}{2}\PY{p}{,} \PY{l+m+mi}{0}\PY{p}{,} \PY{n}{pi}\PY{o}{/}\PY{l+m+mi}{2}\PY{p}{)}\PY{p}{[}\PY{l+m+mi}{0}\PY{p}{]} \PY{o}{*} \PY{n}{p}
\PY{n}{m\PYZus{}2}
\end{Verbatim}
\end{tcolorbox}

            \begin{tcolorbox}[breakable, size=fbox, boxrule=.5pt, pad at break*=1mm, opacityfill=0]
\prompt{Out}{outcolor}{96}{\boxspacing}
\begin{Verbatim}[commandchars=\\\{\}]
1.009779978785009
\end{Verbatim}
\end{tcolorbox}
        
    \begin{tcolorbox}[breakable, size=fbox, boxrule=1pt, pad at break*=1mm,colback=cellbackground, colframe=cellborder]
\prompt{In}{incolor}{97}{\boxspacing}
\begin{Verbatim}[commandchars=\\\{\}]
\PY{c+c1}{\PYZsh{}Дисперсия}
\PY{n}{d} \PY{o}{=} \PY{n}{m\PYZus{}2} \PY{o}{\PYZhy{}} \PY{n}{m} \PY{o}{*}\PY{o}{*} \PY{l+m+mi}{2}
\PY{n}{d}
\end{Verbatim}
\end{tcolorbox}

            \begin{tcolorbox}[breakable, size=fbox, boxrule=.5pt, pad at break*=1mm, opacityfill=0]
\prompt{Out}{outcolor}{97}{\boxspacing}
\begin{Verbatim}[commandchars=\\\{\}]
0.009779978785009025
\end{Verbatim}
\end{tcolorbox}
        
    \begin{tcolorbox}[breakable, size=fbox, boxrule=1pt, pad at break*=1mm,colback=cellbackground, colframe=cellborder]
\prompt{In}{incolor}{ }{\boxspacing}
\begin{Verbatim}[commandchars=\\\{\}]

\end{Verbatim}
\end{tcolorbox}


    % Add a bibliography block to the postdoc
    
    
    
\end{document}
